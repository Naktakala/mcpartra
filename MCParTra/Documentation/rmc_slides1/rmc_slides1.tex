\documentclass{beamer}
\include{amsmath}
\usepackage{amssymb}

\newcommand{\xbold}{\mathbf{x}}

\title{Quick slides on RMC}
\subtitle{}

%\usetheme{lucid}
\begin{document}
	\frame {
		\titlepage
	}
	\frame {
		\frametitle{Basics, sampling from a CDF}
		Given a non-normalized probability distribution, $p^*(\xbold)$, defined over a composite domain $X=V\mathbf{\Omega}$, we can determine a normalized probability distribution, $p(\xbold)$, with
		
		\begin{equation}
		p(\xbold) = \frac{
		p^*(\xbold)
		}{
		\int_X p^*(\xbold)d\xbold
		}.
		\end{equation}
		We can then determine a CDF from
		
		\begin{equation}
		c(\xbold) = \int_{\xbold_0}^\xbold p(\xbold)d\xbold
		\end{equation}
		where $\xbold_0$ is some starting domain point.
	}
	\frame{
		\frametitle{}
		\framesubtitle{}
		We can then determine an inverse CDF, $\xbold_i = c^{-1}(\theta)$, $\theta \in [0,1], \theta \in \mathbb{R}$, from which we can compute the average probability over the sub-domain (or bin), $Y\subset X$, 
		
		\begin{equation}
		p_{Y,avg} = \frac{1}{N} \sum_{i=1}^N
		\begin{cases}
		1, \quad &\text{if } \xbold_i \in Y \\
		0, \quad &\text{if }  \xbold_i \notin Y
		\end{cases}
		\end{equation}
		
		This value integrated over the portion of the composite domain, $A$, can be written as
		
		\begin{equation}
		\int_{\mathbf{\Omega}} p_{Y,avg} d\mathbf{\Omega} = 
		\biggr(
		\int_{\mathbf{\Omega}} d\mathbf{\Omega}
		\biggr)
		\frac{1}{N} \sum_{i=1}^N
				\begin{cases}
				1, \quad &\text{if } \xbold_i \in Y \\
				0, \quad &\text{if }  \xbold_i \notin Y
				\end{cases}
		\end{equation}
		
		Finally:
		\begin{equation}
		\int_{\mathbf{\Omega}} p^*_{Y,avg} d\mathbf{\Omega} =
			\biggr(
			\int_X p^*(\xbold)d\xbold
			\biggr)
			\int_{\mathbf{\Omega}} p_{Y,avg} d\mathbf{\Omega}
		\end{equation}
	
	}
	
	\frame{
	\frametitle{Monte Carlo integration}
	We seek $\int_X p^*(\xbold)d\xbold$ via Monte Carlo integration we can be determined simply from a uniform sampling in $Y$
	
	\begin{equation}
	\int_X p^*(\xbold)d\xbold = \sum_Y \int_Y p^*(\xbold)d\xbold
	\end{equation}
	
	\begin{equation}
	\int_Y p^*(\xbold)d\xbold = 
	\biggr(
	\int_{V_Y}.dV \int_{\mathbf{\Omega}} .d\mathbf{\Omega}
	\biggr)
	\frac{1}{N_Y} \sum_{i=1}^{N_Y} p^*(\xbold_i)
	\end{equation}
	
	}
	
	\frame{
	    \frametitle{Residual}
	    Our residual is given by
	    
	    \begin{equation}
	    r^* = r^*_{interior} + r^*_{surface} 
	    \end{equation}
	    
	    where we lumped surface sources into $r^*_{surface}$. These surface sources present themselves through discontinuities between the present cell flux, $\phi^P$, and the neighboring flux, $\phi^N$. The latter can be a neighboring cell or a boundary. 
	    
	    
	}
	
	\frame{
	\frametitle{Interior residual}
	The interior residual is then given by:
	\begin{equation}
    r^*_{interior} = \frac{1}{4\pi}
   	    \biggr(
    	    q - \sigma_a \phi - \mathbf{\Omega} \cdot \mathbb{\nabla} \phi
    	    \biggr)
    \end{equation}
    from which we can determine the phase-space integration as
    \begin{equation}
    \begin{aligned}
 	\int_X r^*_{interior} .d\xbold
    &=
     \int_V \int_{\mathbf{\Omega}}
     \frac{1}{4\pi}
   	    \biggr(
    	    q - \sigma_a \phi - \mathbf{\Omega} \cdot \mathbb{\nabla} \phi
    	    \biggr)
    	   .d\mathbf{\Omega}.dV \\
    &=  \sum_Y
    	   \int_{V_Y} \int_{\mathbf{\Omega}} r^*_{Y,interior} .d\mathbf{\Omega}.dV \\
    & = \sum_Y  \biggr(
            	\int_{V_Y}.dV \int_{\mathbf{\Omega}} .d\mathbf{\Omega}
            	\biggr)
             r^*_{Y,interior,avg}
    \end{aligned}   
    \end{equation}
    Via Monte Carlo:
    \begin{equation}
    \begin{aligned}
    r^*_{Y,interior,avg} = 
    \frac{1}{N_Y} \sum_{i=1}^{N_Y} r^*_{Y,interior}(\xbold_i)
    \end{aligned}
    \end{equation}

	}
	
	
	\frame{
	\frametitle{Surface residual}
	The surface residual can be presented as
	
	\begin{equation}
		    \int_S \int_{2\pi} \int_{-1}^{0} \mu \ r^*_{surface}.d\mu .d\varphi. dA = 
		    \int_S \int_{2\pi} \int_{-1}^{0} 
		    \mu  \frac{1}{4\pi} (\phi^N - \phi^P)
		    .d\mu .d\varphi. dA
		    \end{equation}
	\begin{equation}
	\begin{aligned}
	\therefore
	\int_X r^*_{surface} .d\xbold &= 
		 \sum_Y  \biggr(
		            	\int_{S_Y}.dS\int_{2\pi} \int_{-1}^{0} \mu.d\mu .d\varphi
		            	\biggr)
		             r^*_{Y,surface,avg} \\
    &=\sum_Y  \sum_f \biggr(
    		            	A_f \pi
    		            	\biggr)
    		             r^*_{Y,f,avg}
	\end{aligned}
	\end{equation}
	Via Monte Carlo:
	\begin{equation}
    \begin{aligned}
    r^*_{Y,f,avg} = 
     \frac{1}{N_f} \sum_{i=1}^{N_f} r^*_{Y,f}(\xbold_i)
     \end{aligned}
 \end{equation}
	}
	
\frame{
\frametitle{Sampling strategy}
Sample the CDF, first interior vs surface:
\begin{itemize}
\item If interior, sample cell ($\to cell \ Y$), sample position randomly in cell, sample direction randomly:
 \begin{equation}
    r^*_{i,interior} = \frac{1}{4\pi}
   	    \biggr(
    	    q - \sigma_a \phi - \mathbf{\Omega} \cdot \mathbb{\nabla} \phi
    	    \biggr)
    \end{equation}
    Particle weight: $w = \frac{r^*_{i,interior}}{r^*_{Y,interior,avg}}$
    \item If surface, sample face ($f$), sample random position on face, sample direction using cosine-law:
  \begin{equation}
  r^*_{i,surface} = \frac{1}{4\pi} (\phi^N - \phi^P)
  \end{equation}  
  Particle weight: $w = \frac{ r^*_{i,surface}}{r^*_{Y,surface,avg}}$
\end{itemize}


Global normalization:
	\begin{equation}
	\int_X r^* .d\xbold = 	\int_X r^*_{interior} .d\xbold + 	\int_X r^*_{surf} .d\xbold
	\end{equation}

}
\end{document}